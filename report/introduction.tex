\chapter{Introduction}

  This report details the implementation and testing of a Mesh Simplification
  algorithm.  The specific algorithm chosen for simplifying the mesh was edge
  collapse, where an edge has its two end points collapsed together into a
  single vertex.

  \section{Edge Collapse}

    Edge collapse is on its face a very simple algorithm, an edge is chosen
    based on a set of error metrics, one end point of the edge is moved halfway
    towards the other then the edges extending from the unmoved end point are
    updated to be connected to the moved end point.  The error metrics chosen
    have a very large effect on how much deformation of the mesh results when
    some fraction of the edges are collapsed, a good set of error metrics will
    initially choose edges that are going to produce the least change in the
    model.  The flip side of this is that the better the error metrics are the
    longer it is likely to take to calculate.

    \subsection{Error Metrics}

      The simple error metric used was mainly based around two key variables.
      The length of the edge and the angle difference between the normals of the
      faces on each side.  The length of the edge is relevant because the longer
      the edge the further the endpoint is going to be moved.  The difference
      between the normals is used as an indicator of the curvature of the local
      area, the greater the angle the more curved it is assumed that the area
      is.  Once these are found they are simply added together.

      The next error metric was based on the one described by Stan Melax
      \cite{smelax} developed for his work at Bioware.  This also has a
      component based on the length of the edge and one based on the local
      curvature, however the local curvature component is based on a larger
      area of the surface to be collapsed.
